% $Id: summary.tex 497 2019-01-29 15:01:30Z smarzani $
%
% Summary and discussions of our results
%------------------------------------------------------------------------
\chapter{Take-home messages and perspectives}\label{summary}

Since many facets and applications of jet substructure have been
covered in this book, it is useless to try and summarise them
all individually.
%
Instead, in this concluding chapter, we will briefly summarise the
main lessons we have learned from about a decade of jet
substructure studies and from the aspects covered in this book.

The first observation is that jet substructure has been a great 
success, both from a theoretical viewpoint and from an
experimental viewpoint. 
%
It took only a few years for the initial idea of
looking at the internal dynamics of jets to grow and develop a myriad
of new tools, opening doors to explore all sorts of new physics
domains.
%
Furthermore, as searches and measurements probe larger and larger
energy scales, boosted-object and jet substructure algorithms are
increasingly relied upon. In particular, at a possible future circular
hadron collider with $\sqrt{s}$ as large as 100~TeV, boosted jets
would be almost omnipresent.

In practice, jet substructure tools are rooted in the theory of strong
interactions. The first generation of substructure techniques were
designed based on core concepts and features of QCD: a QCD jet is
usually made of a single hard core accompanied with soft particles
corresponding to soft-gluon radiation, while boosted massive objects
decay into several hard prongs accompanied by further soft radiation
(at smaller angles if the initial particle is colourless).
%
Key techniques, many of which still in use today, have been developed
starting from these fundamental observations, allowing to establish
jet substructure as a powerful and promising field.
%
A few years later, the introduction of a new generation of
substructure tools was made possible by a better understanding of the
QCD dynamics inside jets using analytic techniques.
%
This first-principle approach has allowed for a more fine-grained
description of the underlying physics, which seeded either simpler
and cleaner tools (e.g.\ the modified MassDropTagger and \SD) or
tools with improved performance (e.g.\ the $D_2$ energy-correlation
functions or dichroic ratios), all under good  theoretical control. 

One of the key features repeatedly appearing when studying jet
substructure from first principles in QCD is the necessity of a
trade-off between performance and robustness. Here, by performance, we
mean the discriminating power of a tool when extracting a given signal
from the QCD background, and by robustness we mean the ability to
describe the tool from perturbative QCD, i.e.\ being as little
sensitive as possible to model-dependent effects such as
hadronisation, the Underlying Event, pileup or detector effects, all
of which likely translate into systematic uncertainties in an
experimental analysis.
%
This trade-off has been seen on multiple occasions throughout this
book.
%
When designing new substructure techniques, we therefore think that it
is helpful to keep in mind both these aspects.

In this context, it was realised that some tools like
\SD or the modified MassDropTagger are amenable to precise
calculations in perturbative QCD, while maintaining small
hadronisation and Underlying Event corrections.
%
This is particularly interesting since jet substructure tools are
often sensitive to a wide range of scales --- between the TeV scale
down to non-perturbative scales --- offering an almost unique
laboratory for QCD studies.
%
It has opened new avenues for future jet
substructure studies. 
%
 A typical example is a potential for
an extraction of the strong coupling constant from substructure
measurements (see e.g.\ Ref.~\cite{Bendavid:2018nar}), but other
options include the improvement of Monte Carlo parton showers,
measurements of the top mass, or simply a better control over QCD
background for new physics searches.

Because of its potential for interesting Standard Model measurements
across a wide range of scales, jet substructure has also recently
found applications in heavy-ion collisions.
%
One of the approaches to study the quark-gluon plasma is by analysing
how high-energy objects are affected by their propagation through it.
%
The LHC is the first collider where jets are routinely used for this
type of studies and an increasing interest for jet substructure
observables has been seen very recently in the heavy-ion community.
%
This will for sure be an important avenue in the future of jet
substructure, including the development of specific observables to
constraint the properties of the quark-gluon plasma and their study in
QCD.

The analysis of cosmic ray interactions is a further area of research
where jet substructure techniques were introduced to study the
detailed structure of complicated objects \cite{Brooijmans:2016lfv,
  Aab:2018jpg}. Ultra-high-energy cosmic rays, e.g. protons, can
produce interactions with very high momentum transfer between when
they scatter of atoms of Earth's atmosphere. Such interactions produce
a collimated high-multiplicity shower of electrons, photons and
muons. Their spacial distribution and penetration depth can be
analysed to inform the nature of the incident particle and interaction
in the collision. It is likely that in the near future, with the
increased interest in so-called beam-dump experiments, more ideas are
going to be introduced where jet substructure techniques can become of
importance.

Finally, one should also expect the future to deliver its fair share
of new tools for searches and measurements.
%
We believe that there are two emblematic directions worth exploring.
%
An obvious direction is the one of machine-learning tools. This is an
increasingly hot topic in the jet substructure community and one
should expect it to continue growing in importance.
%
In the context of the first-principle understanding used throughout
this book, one should highlight that it is important to keep in mind
that applying machine-learning techniques to jet substructure is not
just a problem for computer scientists. These algorithms are to a large
extent dealing with QCD and therefore a good control of the QCD
aspects of jet substructure is crucial. Several examples of this have
appeared very recently --- like QCD-aware
networks~\cite{Louppe:2017ipp}, energy-flow polynomials and
networks~\cite{Komiske:2017aww,Komiske:2018cqr} or the Lund jet
plane~\cite{Dreyer:2018nbf} --- and we should definitely expect more
in the future.
%
One can even imagine to extend concepts developed for jets to be
applied to the full event, i.e. a full-information approach to study
the whole radiation profile of an event. This could maximise the
sensitivity of collider experiments in searches for new physics.

The second direction we want to advocate for is the development of
additional tools which are theory-friendly, i.e.\ that are under
analytical control and are amenable for precision calculations.
%
As shown in this book, basic substructure tools have now been
understood from first-principles, including the main physics aspects
responsible for the trade-off between performance and
robustness. However, modern boosted jet taggers involve several of
these tools in order to maximise performance (cf.\ our discussion in
chapter~\ref{tools}). 

We think that new tools offering a combination of grooming,
prong-finding and radiation constraints will always be of great value.
%
Compared to a deep-learning-based tool, this
might show a small loss in performance, but it would offer the
advantage of a better control of its behaviour across a wide range of
processes and studies. One of the key ingredients here is that these
new tools should remain as simple as possible to facilitate their
calibration in an experimental context, hopefully resulting in small
systematic uncertainties. This would make them usable for the
precision programme at the LHC, including both measurements and
searches.
%
From an analytic perspective, achieving precision for such
substructure algorithms will also require further developments in
resummation techniques and fixed-order (amplitude) calculations, where
many promising results have already been obtained recently. 

% extensions to other fields

All this being said, we hope that we have conveyed the idea that jet
substructure has been a fascinating field for almost a decade, with an
ever-growing range of applications. Over this time-span, the field has
managed to stay open to new ideas and new approaches. One should
therefore expect more exciting progress in the years to come.
%
We therefore hope that this book gives a decent picture of the state
of the field in early 2019 and will constitute a good introduction
for newcomers to the field.

\vspace{0.5cm}

\begin{center}
\emph{If you ain't boostin' you ain't living}\\
\emph{\textexclamdown Boostamos!}~\cite{boostamos}
\end{center}



%% GS helper for auctex
%%% Local Variables:
%%% mode: latex
%%% TeX-master: "notes"
%%% End:
