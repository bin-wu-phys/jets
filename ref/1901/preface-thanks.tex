\chapter*{Preface}
The study of the internal structure of hadronic jets has become in recent years a very active area of research in particle physics. Jet substructure techniques are increasingly used in experimental analyses by the Large Hadron Collider collaborations, both in the context of searching for new physics and for Standard Model measurements. 
On the theory side, the quest for a deeper understanding of jet
substructure algorithms has contributed to a renewed interest in
all-order calculations in Quantum Chromodynamics (QCD).
%
% , attracting new people to the field, which
This has resulted in new ideas about how to design better observables
and how to provide a solid theoretical description for them.
%
In the last years, jet substructure has seen its scope extended, for
example, with an increasing impact in the study of heavy-ion
collisions, or with the exploration of deep-learning techniques.
%
Furthermore, jet physics is an area
% of research in particle physics
in which experimental and theoretical approaches meet together, where cross-pollination and collaboration between the two communities often bear the fruits of innovative techniques.
%
The vivacity of the field is testified, for instance, by the very successful series of BOOST conferences together with their workshop reports, which constitute a valuable picture of the status of the field at any given time. 

However, despite the wealth of literature on this topic, we feel that a comprehensive and, at the same time, pedagogical introduction to jet substructure is still missing. This makes the endeavour of approaching the field particularly hard, as newcomers have to digest an increasing number of substructure algorithms and techniques, too often characterised by opaque terminology and jargon. Furthermore, while first-principle calculations in QCD have successfully been applied in order to understand and characterise the substructure of jets, they often make use of calculational techniques, such as resummation, which are not the usual textbook material.
%
This seeded the idea of combining our experience in different aspects of jet substructure phenomenology to put together this set of lecture notes, which we hope could help and guide someone who moves their first steps in the physics of jet substructure.  

\pagebreak

\centerline{\bf Acknowledgements}
\vspace*{0.5cm}

Most of (if not all) the material collected in this book comes from years of collaboration and discussions with excellent colleagues that helped us and influenced us tremendously. In strict alphabetical order, we wish to thank %\sm{please add!}
Jon Butterworth,
Matteo Cacciari,
Mrinal Dasgupta,
Frederic Dreyer,
Danilo Ferreira de Lima,
Steve Ellis,
Deepak Kar,
Roman Kogler,
Phil Harris,
Andrew Larkoski,
Peter Loch,
David Miller,
Ian Moult,
Ben Nachman,
Tilman Plehn,
Sal Rappoccio,
Gavin Salam,
Lais Schunk,
Dave Soper,
Michihisa Takeuchi,
Jesse Thaler, and
Nhan Tran.
%
We would also like to thank Frederic Dreyer, Andrew Lifson, Ben
Nachman, Davide Napoletano, Gavin Salam and Jesse Thaler for helpful suggestions and
comments on the manuscript.

%  LocalWords:  calculational Butterworth Mrinal Dasgupta Larkoski
%  LocalWords:  Loch Nachmann Tilman Plehn Rappoccio Soper Thaler
%  LocalWords:  Nhan
